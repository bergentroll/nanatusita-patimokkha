(Vinayapucchāsammuti:) Namo tassa bhagavato arahato sammā sambuddhassa. (3x)

(Ñatti:) Suṇātu me, bhante, saṅgho. Yadi saṅghassa pattakallaṃ, ahaṃ itthannāmaṃ bhikkhuṃ (i.e., āyasmantaṃ…) vinayaṃ puccheyyaṃ.

(Vinayavissajjanasammuti:) Namo tassa bhagavato arahato sammāsambuddhassa.

(Ñatti:) Suṇātu me bhante saṅgho. Yadi saṅghassa pattakallaṃ, ahaṃ itthan-nāmena bhikkhunā [i.e., āyasmatā …] vinayaṃ puṭṭho vissajjeyyaṃ.

(Pubbakaraṇa-pucchā-vissajjanā:) (Pucchako:) Sammajjanī padīpo ca, udakaṃ āsanena ca,  uposathassa etāni pubbakaraṇan’ti vuccati. Okāsa, sammajjanī: Sammajjanakaraṇaṃ kataṃ kiṃ?

(Vissajjako:) Sammajjanakaraṇaṃ niṭṭhitaṃ.

(Puc.:) Padīpo ca: padīpujjalanaṃ kataṃ kiṃ?

(Vis.:)Padīpujjalanaṃ niṭṭhitaṃ. [or:] Idāni pana suriy'ālokassa atthitāya
padīpakiccaṃ idha n’atthi.

(Puc.:)Udakaṃ āsanena ca: Āsanena saha pānīyaparibhojanīya-
udakaṭṭhapanaṃ kataṃ kiṃ?

(Vis.:) Āsanena saha pānīyaparibhojanīya-udakaṭṭhapanaṃ niṭṭhitaṃ.

(Puc.:) Uposathassa etāni pubbakaraṇan’ti vuccati kiṃ?

(Vis.:) Etāni cattāri vattāni sammajjanakaraṇādīni saṅghasannipātato
paṭhamaṃ kattabbattā, uposathassa uposathakammassa pubbakaraṇan-ti
vuccati. Pubbakaraṇānī'ti akkhātāni.

(Pubbakicca-pucchā-vissajjanā:)2

(Puc.:) Chandapārisuddhi utu'kkhānaṃ, bhikkhugaṇanā ca ovādo,
uposathassa etāni pubbakiccan-ti vuccati.
Chandapārisuddhi: Chandārahānaṃ bhikkhūnaṃ chandapārisuddhi-
āharaṇaṃ kataṃ kiṃ?

(Vis.:) Chandapārisuddhi-āharaṇaṃ niṭṭhitaṃ. (or:) Idha n’atthi.

(Puc.:) Utu'kkhānaṃ: Hemantādīnaṃ tiṇṇaṃ utūnaṃ ettakaṃ
atikkantaṃ ettakaṃ avasiṭṭhan’ti. Evaṃ utu-ācikkhanaṃ kataṃ kiṃ?

(Vis.:) Utūn'īdha pana sāsane hemanta-gimha-vassānānaṃ vasena tīṇi
honti. Ayaṃ hemanta-/gimha-/vassāna-utu. Asmiṃ utumhi aṭṭha (dasa)
uposathā. Iminā pakkhena eko uposatho sampatto, … uposatho/ā
atikkanto/ā, … uposathā avasiṭṭho/ā.

(Puc.:)Bhikkhugaṇanā ca: Imasmiṃ uposath'agge sannipatitānaṃ
bhikkhūnaṃ gaṇanā, kittakā bhikkhū honti?

(Vis.:)Asmiṃ uposath'agge sannipatitānaṃ bhikkhūnaṃ gaṇanā cattāro/
pañca … bhikkhū honti.

(Puc.:) Ovādo: Bhikkhunīnaṃ ovādo dātabbo dinno kiṃ?

(Vis.:) Idāni pana tāsaṃ n’atthitāya, so ca ovādo idha n’atthi.

(Puc.:) Uposathassa etāni pubbakiccan’ti vuccati kiṃ?

(Vis.:) Etāni pañcakammāni chand'āharaṇādīni pātimokkhuddesato
paṭhamaṃ kattabbattā, uposathassa uposathakammassa pubbakiccan-ti
vuccati. Pubbakiccānī'ti akkhātāni.

(Pattakalla-pucchā-vissajjanā:)2

(Puc.:) Uposatho yāvatikā ca bhikkhū kammappattā sabhāg'āpattiyo ca na
vijjanti, vajjanīyā ca puggalā tasmiṃ na honti pattakallan-ti vuccati.
Uposatho: Tīsu uposathadivasesu cātuddasī paṇṇarasī, sāmaggīsu,
ajj’uposatho ko uposatho?

(Vis.:) Ajj’uposatho cātuddaso/paṇṇaraso.

(Puc.:) Yāvatikā ca bhikkhū kammappattā'ti kiṃ?

(Vis.:) Yattakā bhikkhū tassa uposathakammassa pattā, yuttā, anurūpā,
sabbantimena paricchedena cattāro bhikkhū pakatattā, saṅghena
anukkhittā, te ca kho hatthapāsaṃ avijahitvā ekasīmāyaṃ ṭhitā.

(Puc.:) Sabhāg'āpattiyo ca na vijjanti kiṃ?

(Vis.:) vikālabhojan'ādi- vatthu sabhāg'āpattiyo ca na vijjanti.

(Puc.:) Vajjanīyā ca puggalā tasmiṃ na honti kiṃ?

(Vis.:) Gahaṭṭha-paṇḍakādayo, ekavīsati vajjanīyā puggalā, hatthapāsato
bahikaraṇavasena vajjetabbā. Te asmiṃ na honti.

(Puc.:) Pattakallan-ti vuccati kiṃ?

(Vis.:) Saṅghassa uposathakammaṃ imehi catūhi lakkhaṇehi saṅgahitaṃ
pattakallan-ti vuccati: Pattakālavantan892–ti akkhātaṃ.

(Árádhana:)
(Vis.:) Pubbakaraṇapubbakiccāni samāpetvā desitāpattikassa samaggassa
bhikkhusaṅghassa anumatiyā pātimokkhaṃ uddisituṃ ārādhanaṃ
karomi.

\clearpage

(The authorisation for asking about the Vinaya.)

[Q:] Homage to the Fortunate One, the Worthy One, the Perfectly
Awakened One. (3x)

(Announcement)
Venerable Sir, please let the Community listen to me! If it is suitable
to the community, (then) I would ask the such-named venerable
about the Discipline.

(The authorisation to answer with regards the Vinaya.)
[A:] Homage to the Fortunate One, the Worthy One, the perfectly
Awakened One. (3x)

(Announcement)
Venerable Sir, please let the Community listen to me! If it is suitable
to the community (then) I would answer having been asked about the
(The questioning and answering with regards the preparations.)

[Q:] “The broom and the lamp, the water with the seat these are
called “the preparation for the observance.”
Permit [me to ask]! The broom. Has the action of sweeping been done?

[A:] The action of sweeping is finished.

[Q:] And the lamp. Has the lighting of the lamp been done?

[A:] The lighting of the lamp is finished. (Or:) There is no lamp-duty
because of the state of there being sunlight now.

[Q:] The water with the seat. Has the setting up of the drinking- and
washing-water been done?

[A:] The placing of the drinking- and washing-water together with the
seat is finished.

[Q:] Are these called “the preparation for the observance”?

[A:] These four protocols, the action of sweeping, etc., due to the
having to be done first before the gathering of the Community are
called “the preparation for the [legal] act of Uposatha on the Uposatha.”

The “preparations” have been announced.
 (The questioning and anwering with regards the preliminary duties.)

[Q:] The consent and purity, the telling of the season, the counting of
the bhikkhus and the instruction (of the bhikkhunīs), these are called:
“the preliminary duty for the observance.”

[Q:] The consent and purity. Has the bringing of consent [and] purity
of bhikkhus who are deserving of [giving] consent been done?

[A:] The bringing of consent [and] purity of bhikkhus who are
deserving of [giving] consent is finished. (Or:) [There] is no [bringing
of consent and purity] here.

[Q:] The telling of the season. “Of the three seasons, the winter, etc.,
this number [of Uposathas] have passed, this number [of Uposathas]
are left,” has the telling of the season been done thus?

[A:] In this dispensation there are three seasons, [namely] the winter,
the summer, and the rains.
This is the winter-/summer-/rainy-season, and in this season there are
eight (ten) Uposathas. With this fortnight (one) Uposatha has arrived,
one/two/ … Uposatha(s) is/have past, one/two/… Uposatha(s) is/are
left.

[Q:] The counting of the bhikkhus. The counting of the bhikkhus
who have gathered in this Uposatha-hall is the counting of the

[Q:] The instruction. The instruction to be given to the bhikkhunīs:
has it been given?

[A:] Because of their non-existence now, there is no instruction here.

[Q:] Are these called “the preparation for the observance”?

[A:] These four protocols, the bringing of consent, etc., due to the
having to be done first before the gathering of the Community are
called “the preliminary duty for the [legal] act of Uposatha on the
Uposatha.”

The “preliminary duties” have been announced.
(The questioning and answering about the reached suitability.)

[Q:] The observance, whatever bhikkhus are entitled (to carry out the
legal) act, common offences are not found, there are no persons to be
excluded in there, this is called: “reached suitability`.”

[Q:] The Uposatha. With regards the fourteenth [-day Uposathas], the
fifteenth [-day Uposathas], or the Unity [-Uposathas], the Uposatha
today is which Uposatha?

[A:] Today is the fourteenth/fifteenth day Uposatha.

[Q:] What is “Whatever bhikkhus are entitled (to carry out the legal)
act”?

[A:] However many bhikkhus who are entitled to that Uposatha-act,
who are fit, are suited, with the minimum amount (of bhikkhus), four
bhikkhus who are regular, not suspended by the community, and
they, not having left arms-length, remain in the same boundary.

[Q:] Are common offences not found?

[A:] Common offences, founded on eating at the wrong time and so
on, are not found.

[Q:] Are there in that [arms-length] no persons fit to be excluded

[A:] Householders, eunuchs, etc., the twenty-one persons fit to be
excluded, who are is to be excluded by making [them go] outside from
the arms-length, they are not in that [arms-length.]

[Q:] What is said to “have reached suitability”?

[A:] The [legal] act of Uposatha endowed with these four
characteristics is called “[one that] has reached suitability.” “[It] has
reached the time” is said.
(Invitation)

[A:] “Having concluded the preparations and preliminary duties I
make the invitation to recite the Disciplinary Code with the approval
