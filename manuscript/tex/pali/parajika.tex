\section{Pārājik'uddeso}
\label{par}

\begin{intro}
  Tatr'ime cattāro pārājikā dhammā uddesaṁ āgacchanti.
\end{intro}

\setsubsecheadstyle{\subsubsectionFmt}
\pdfbookmark[2]{Pārājika 1}{par1}
\subsection*{\hyperref[disq1]{Pārājika 1: Methunadhammasikkhāpadaṁ}}
\label{par1}
Yo pana bhikkhu bhikkhūnaṁ sikkhāsājīvasamāpanno sikkhaṁ appaccakkhāya dubbalyaṁ anāvikatvā methunaṁ dhammaṁ paṭiseveyya, antamaso tiracchānagatāya'pi; pārājiko hoti, asaṁvāso.

\pdfbookmark[2]{Pārājika 2}{par2}
\subsection*{\hyperref[disq2]{Pārājika 2: Adinn'ādānasikkhāpadaṁ}}
\label{par2}
Yo pana bhikkhu gāmā vā araññā vā adinnaṁ theyyasaṅkhātaṁ ādiyeyya, yathārūpe adinn'ādāne rājāno coraṁ gahetvā haneyyuṁ vā bandheyyuṁ vā pabbājeyyuṁ vā: ``Coro'si, bālo'si, mūḷho'si, theno'sī'ti,'' tathārūpaṁ bhikkhu adinnaṁ ādiyamāno; ayam'pi pārājiko hoti, asaṁvāso.

\pdfbookmark[2]{Pārājika 3}{par3}
\subsection*{\hyperref[disq3]{Pārājika 3: Manussaviggahasikkhāpadaṁ}}
\label{par3}
Yo pana bhikkhu sañcicca manussaviggahaṁ jīvitā voropeyya, satthahārakaṁ v'āssa pariyeseyya, maraṇavaṇṇaṁ vā saṁvaṇṇeyya, maraṇāya vā samādapeyya: ``Ambho purisa, kiṁ tuyh'iminā pāpakena dujjīvitena? Matan'te jīvitā seyyo'ti!'', iti cittamano cittasaṅkappo anekapariyāyena maraṇavaṇṇaṁ vā saṁvaṇṇeyya, maraṇāya vā samādapeyya; ayam'pi pārājiko hoti, asaṁvāso.

\pdfbookmark[2]{Pārājika 4}{par4}
\subsection*{\hyperref[disq4]{Pārājika 4: Uttarimanussadhammasikkhāpadaṁ}}
\label{par4}
Yo pana bhikkhu anabhijānaṁ uttarimanussadhammaṁ att'ūpanāyikaṁ alam'ariyañāṇadassanaṁ samudācareyya: ``Iti jānāmi, iti passāmī'ti!'', tato aparena samayena samanuggāhiyamāno vā asamanuggāhiyamāno vā āpanno visuddh'āpekkho evaṁ vadeyya: ``Ajānam'ev'āhaṁ āvuso avacaṁ: 'Jānāmi!' apassaṁ: 'Passāmi!' Tucchaṁ musā vilapin'ti'', aññatra adhimānā, ayam'pi pārājiko hoti, asaṁvāso.

\medskip

\begin{center}
Uddiṭṭhā kho āyasmanto cattāro pārājikā dhammā. Yesaṁ bhikkhu aññataraṁ vā aññataraṁ vā āpajjitvā na labhati bhikkhūhi saddhiṁ saṁvāsaṁ. Yathā pure, tathā pacchā, pārājiko hoti, asaṁvāso.

\smallskip

Tatth'āyasmante pucchāmi: Kacci'ttha parisuddhā?\\
Dutiyam'pi pucchāmi: Kacci'ttha parisuddhā?\\
Tatiyam'pi pucchāmi: Kacci'ttha parisuddhā?

\smallskip

Parisuddh'etth'āyasmanto, tasmā tuṇhī, evam'etaṁ dhārayāmi.
\end{center}

\linkdest{endnote10-body}
\begin{outro}
  Cattāro pārājkā niṭṭhitā\makeatletter\hyperlink{endnote10-appendix}\Hy@raisedlink{\hypertarget{endnote10-body}{}{\pagenote{%
    \hypertarget{endnote10-appendix}{\hyperlink{endnote10-body}{Not in any edition or manuscript, but if a conclusion is to be recited then this one as given in the Parivāra would be the suitable one.\\
        When reciting in brief use: pārājik'uddeso niṭṭhito.}}}}}\makeatother
\end{outro}

\clearpage
